\chapter{The LHC Machine}

The famous quote "history repeats itself" applies well to the \acrfull{hep}. The starting point of experimental high energy physics is the Rutherford $\alpha$-particle scattering, and even now we are doing the same thing just the method changed from "natural accelerator" to the "man-made" accelerator that can accelerate particles with the velocity close to the speed of light. The design and working of accelerator changed a lot over a period in going from MeV to GeV and now to the multi-TeV range. Now, these machines are not only used in \acrshort{hep} experiments, but it went to treat human beings like cancer therapy, radioisotope production,  to the industry for uses like material processing, sterilisation, security scan, water treatment, and many more. 



Below table is taken from \cite{Schoerner-Sadenius2015, LHC-parameters-2016, LHC-tdr}. THis is ram.


\begin{tabularx}{\textwidth}{|l c c|}
    \hhline{---}
    {\bf } \cellcolor[gray]{.8}& {\bf Injection} \cellcolor[gray]{.8}& {\bf Collision}\cellcolor[gray]{.8} \\
    \hhline{---}
    {\bf Proton energy (GeV)} & 450 & 6500\\
    \hhline{---}
    {\bf Circumference (m)} &  26658.883 & \\
    \hhline{---}
    {\bf Particles/bunch ($10^{11}$)}& 1.18 &  \\
    \hhline{---}
    {\bf Number of bunches} &   2076 &  \\
    \hhline{---}
    {\bf Bunch distance (ns)}   &  25 &   \\
    \hhline{---}
    {\bf Bunch length (ns)} &    1.05 & \\
    \hhline{---}
    {\bf Beam current (mA)} &   584 &   \\
    \hhline{---}
    {\bf Norm. emittance (x and y) ($\mu$ mrad)}  &  3.5 &  2.6 \\
    \hhline{---}
    {\bf Stored energy per beam (MJ)}   &  23.3 &  362 \\
    \hhline{---}
    {\bf Rms beam size at IP1 and IP5 ($\mu$ m)}  &  375 &  17 \\
    \hhline{---}
    {\bf Rms beam size at IP2 and IP8 ($\mu$ m)}  &  280 &  71 \\
    \hhline{---}
    {\bf Peak luminosity ($cm^{-2}s^{-1}$)}   &   & 1.1 $\times ~ 10^{34}$  \\
    \hhline{---}
    {\bf Average luminosity lifetime ($\tau$) (hours)} &    &   24 \\
    \hhline{---}
    {\bf Average mean pile-up}  &   &   27  \\
    \hhline{---}
\end{tabularx}
% Ref: \url{https://lhc-commissioning.web.cern.ch/lhc-commissioning/performance/2016-performance.htm]}

