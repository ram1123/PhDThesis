\chapter{Motivation: Gas Electron Multiplier}
The Compact Muon Solenoid (CMS) is a general purpose detector designed to optimize the discovery potential of the LHC collider. As the name suggest, detecting muons is one of its most important tasks because they are unmistakable signature of most of the Physics. So, The ability to trigger on and reconstruct muons at the highest luminosities is central to the concept of CMS.

Muons are charged particles that are just like electrons and positrons, but are 200 times heavier. We expect them to be produced in the decay of a number of potential new particles; for instance, one of the clearest ``signatures" of the Higgs Boson is its decay into four muons.

Goal of the CMS muon system is muon identification, momentum measurement and trigger. It is composed of three detection technologies:  the Drift Tubes (DT), the Cathode Strips Chambers (CSC) and the Resistive Plate Chambers (RPC). Precision measurement are provided by Drift Tubes (DT) in the barrel, covering acceptances up to $|\eta|<1.2$, and Cathode Strip Chambers (CSC) in the endcaps covering $1.0<|\eta|<2.4$. Resistive Plate Chambers (RPC) ensure adequate redundancy and triggering up to $|\eta|<1.6$, but the originally-planned redundancy was not implemented beyond $|\eta|>1.6$ where the background particles rates are highest and the bending in the magnetic field is the smallest. But, most events of interest will have one or more muons at higher rapidity so the muon endcaps are of equal importance.

So, As the LHC will pushes towards both energy and luminosity to open up discovery potentials, the quality of muons detection will play important role for discovery. Thus, we need to upgrade this region and re-establish the originally forseen redundancy in the forward region beyond $\eta$ 1.6 based on modern, high-resolution and fast gas detectors capable to operate up to MHz rates.

\section{Why Muons so important?}
{\bf Reference: http://mu2e.fnal.gov/why-muons.shtml \\ http://cms.web.cern.ch/news/muon-detectors\\http://nmi3.eu/muon-research/science.html\\http://profmattstrassler.com/2013/01/31/the-puzzle-of-the-proton-and-the-muon/\\http://profmattstrassler.com/2013/01/31/the-puzzle-of-the-proton-and-the-muon/}
Particle physics has been very successful in creating the Standard Model, a theoretical framework that describes all known particles and their interactions. But this model appears to be incomplete and breaks down at high energies such as those that existed shortly after the Big Bang. This suggests new particles and interactions exist beyond what is accounted for in the Standard Model.

%One mystery in this catalogue of particles and forces is that in the framework of the Standard Model, particles fit neatly into three categories with similar properties, yet very different masses. The heaviest of these mass states have not existed naturally since shortly after the Big Bang. Physicists theorize that a rare, undiscovered combination of forces interacting on particles could cause this mass variation. This unification of all known forces — the gravitational, electromagnetic, weak and strong forces - could hold sway over our world.

Mu2e will help test whether what is at work is a merging of all forces into one "grand unification" of forces that existed shortly after the Big Bang when the universe was full of heavy particles. A muon that does not follow the traditional weak-force decay pattern into a lighter electron and two neutrinos, but converts wholly into an electron would signal the existence of new particles or new forces - particles or forces beyond those included in the Standard Model. These new particles or forces would play a role in the unification process.

For efficiency reasons, physicists chose to study the conversion of a muon, the second heaviest building block of matter, and not the heavier tau particle's conversion. Muons live longer before they decay into lighter particles and are easier to produce in large quantities.

Seeing muon-to-electron conversion will be more difficult than finding a needle in a haystack and will require patience and vast quantities of muons.

In fact, theorists predict that this type of conversion happens so rarely that observing it would equate to finding one penny with a unique scratch on Abe Lincoln's head. That penny would be hidden in one of 234 piles of pristine pennies with each pile amounting to the 2010 U.S. budget of \$3.55 trillion.

To increase the probability of seeing this rare subatomic process, physicists will generate huge numbers of muons by colliding a proton beam with a target.

During the experiment's initial two-year running period, it will produce about one quintillion muons, which is roughly the number of grains of sand on Earth's beaches.

This research strategy of producing vast quantities, or intense amounts, of particles rather than focusing on accelerating them to the highest energy marks studies on the Intensity Frontier. Particle physics research is divided into three broad approaches called frontiers. Studies on the Intensity Frontier require extreme machines, such as multi-megawatt proton accelerators that produce high-intensity beams, which generate large numbers of particles.

A second-phase, upgraded Mu2e experiment could utilize a proposed high-intensity upgrade to Fermilab's proton accelerator, that would increase the production of muons by one to two orders of magnitude.

{\bf A STUDY OF MUONS CAN ANSWER THE FOLLOWING QUESTIONS:}

Muons could provide the path to unveiling hidden physics phenomena. In particular, physicists hope that muons will shed light on these questions:

Are there undiscovered principles of nature: new symmetries, new physical laws?
Do all forces that dictate particle interactions merge into one force, called a grand unification, at higher energy scales?
Why are there so many kinds of particles?
If charged lepton flavor violation occurs, is it related to the flavor violation seen with quarks? Or is it due to new phenomena at the Terascale, an energy region named for the tera, or million million, electronvolts of energy needed to access it?
Physicists can study muons and learn more about their role in the universe by making huge numbers of them with a high-intensity accelerator and by building a set of magnets that increases the chance of observing muon-to-electron conversion. The proposed Mu2e experiment will provide unprecedented sensitivity to this conversion process and will offer a new window on physics beyond the Terascale, including unification.

Learn more about how Mu2e would greatly advance research at the Intensity Frontier in a report by the U.S. Particle Physics Prioritization Panel, P5.
