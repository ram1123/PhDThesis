\chapter{Introduction}

Particle physics is a modern name for the centuries old effort to understand
the basic laws of physics. -- Edward Witten


Everybody is excited to know about the behaviour of nature around us, e.g. we are standing on earth but where the earth is standing, how it comes into existence, why there are lots of stars, and so on. These are the general questions asked by every human being, and the fundamental physics provides the answer. If we would like to understand it deeper and deeper, it will lead us to the particle physics. As the name infers it deals with the particles (the fundamental particles). However, the question is that how these fundamental particles will give insight into our universe? The answer is that in early universe there are only particles and if we go near the event horizon, then there was only quark gluons plasma. To verify these things, we cannot go back to the early universe, but we can probe them using the high energy colliders to know their behaviour.

One of best model that explains the world around us is the Standard Model (SM). With the discovery of the Higgs boson \cite{Chatrchyan:2012xdj} in 2012 it was complete. However, even now many questions remain unanswered by the SM. List of few questions are:

\begin{itemize}
\item How neutrino get its mass and what is its mass hierarchy? 
\item Why there are only three generations of leptons and quarks? 
\item Why the third generation is too much heavy as compared to other two generations?
\end{itemize}

Now there are two ways to answer these questions. First is to go for another new theory or a model like SUSY, string theory, and others. Another way is to understand the SM profoundly and look for the possible extensions of SM. One of the possible ways is to look for WW scattering. As the Higgs boson unitrize the WW scattering amplitude, so its one of the best path to look for SM deviation.

\begin{itemize}
\item Non-abelian nature => QGC in addition of trilinear gauge coupling.
\item SM includes three types of GGC.
    \item $W^+W^-W^+W^-$
    \item $W^+W^-ZZ$
    \item $W^+W^-\gamma \gamma$
\item ZZZZ vertex is not persent in SM but it is present at tree level via exchange of Higgs boson.
\item $\gamma \gamma ZZ$ vertex is only produced at loop level in SM.
\item Trilinear and quartic couplings probe different aspect of the weak interactions.
    \item Trilinear coupling test the non-abelian nature of SM
    \item Quartic coupling probe the EWSB.
\end{itemize}

\begin{itemize}
\item Heavy vector boson $W^{\pm}$ and $Z$ acquire there mass and longitudinal polarization state through spontaneous EWSB.
\item The mechanism for EWSB must regulate $\sigma(V_LV_L \rightarrow V_LV_L )$ to restore unitarity above $\approx$ 1-2 TeV.
    \item cross-section attenuated to a linear growth by the quartic gauge by the quartic gauge self coupling.
    \item a light SM higgs boson exactly cancels increase for large s (for HWW coupling).
\end{itemize}

\section{Outline and Contributions}


Part-1 of this thesis is dedicated to the CMS detector.

Part-2 consists of my GEM work.

Part-3 consists of physics analysis.

\begin{itemize}
\item Introduction and motivation     
\item What is particle physics?     
\item Introduction to particle physics?
\item Introduction to GEM detectors 
\item Motivation for HEP detectors
\item LHC and CMS 
\item GEM detectors 
\item GEM Hardware work
\item GEM test beam work
\item Physics of WW scattering  
\item Physics Analysis details     
\item Basic experimental techniques     
\item MC generation and study     
\item Data/MC comparison study    
\item Background estimation     * TMVA     * Limits for aQGC and Charged Higgs *
\item Summary and outlook
\end{itemize}

\section{Summary of chapters}

This is an outline of what went into each chapter. **Chapter 1** gives a
background on duis tempus justo quis arcu consectetur sollicitudin.  **Chapter
2** discusses morbi sollicitudin gravida tellus in maximus.  **Chapter 3**
discusses vestibulum eleifend turpis id turpis sollicitudin aliquet.  **Chapter
4** shows how phasellus gravida non ex id aliquet. Proin faucibus nibh sit amet
augue blandit varius.
