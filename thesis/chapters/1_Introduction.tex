\chapter{Introduction}
\begin{chapquote}
{Edward Witten}
``Particle physics is a modern name for the centuries old effort to understand the basic laws of physics.''
\end{chapquote}
%
% Mandatory citations: 
% aQGC thesis by Carsten Bittrich ~\cite{Bittrich2012}
%

Everybody is excited to know about the behaviour of nature around us, e.g. we are standing on earth but where the earth is standing, how it comes into existence, why there are lots of stars, and so on.
These are the general questions asked by every human being. Answers and explanations to all these question at deeper level will lead us to the one of branch of fundamental physics known as ``\textit{particle physics}''.
% If we would like to understand it deeper and deeper, it will lead us to the particle physics.
As the name infers it deals with the particles (the fundamental particles) of matter and the interactions among them.

In the last century a theory was developed to describe the fundamental particles and their interactions and is known as the Standard Model (SM) of particle physics.
Up to now this theory is well tested and with the discovery of the Higgs boson~\cite{Chatrchyan:2012xdj,Aad:2012tfa} the spectrum of particles predicted by it is complete.
Now the next important task is to study rigorously the precision measurement and test the deviation in the SM prediction, specially in the electroweak sector~\cite{Baak2013}.

% In this chapter, first a brief introduction to SM is given then a unique way to explore the electroweak sector using the anomalous quartic gauge coupling is discussed probed by the vector boson scattering. Also, the doubly charged Higgs model, George-Machak model, is discussed. It gives multiple Higgs candidate using Higgs triplets by preserving the custodial symmetry. And in return it gives neutrino a Majorana mass.

As the electroweak gauge bosons $W^{\pm}$, $Z$ and $\gamma$, through mixing, represents the SM sector coupled strongly to the Electroweak Symmetry Breaking (EWSB).
And the Higgs discovery give faith that the ultimate EWSB should be much like the simple Higgs mechanism.
As the electroweak gauge bosons offer a relatively clean signal at the Large Hadron Collider (LHC), if there is any hidden structure to the EWSB beyond the Higgs then this is one of the best way to look for this~\cite{Green2017}.

In this thesis the study of one of such channel where an interaction vertices have four bosons, known as quartic gauge coupling, is studied through vector boson scattering process.
Also a model interpretation is done with George-Machak model which gives us doubly charged Higgs.
Further as we know that to perform these studies at very fundamental level we need a very sophisticated accelerator to accelerate the particles and to collide them and a detector to capture the remnants.
Only this way one can perform the studies in experimental particle physics.

The one of such accelerator, LHC, and its one of main purpose detector Compact Muon Solenoid (CMS) is described in chapter~\ref{cha:the_lhc_and_cms_machine}, while the brief introduction of the SM and the Vector Bososn Scattering (VBS) is explained from experimental point of view in chapter~\ref{cha:standard_model__vector_boson_scattering}. Also with time and to look for rare process along with to increase the production rate of the existing process such as vector boson scattering with time and latest available technology we should upgrade our detector. In this thesis the performance of one of such detector, Gas Electron Multiplier (GEM), is described in chapter~\ref{cha:gas_electron_multiplier}. This detector is going to be installed during Long Shutdown-2 (2019-2020) in CMS detector to improve the triggering and tracking of muons in higher pseudo-rapidity region. Also, in the same chapter the characterization of one of basic ingredients of GEM detector, known as GEM foil, is described which is fabricated by an Indian company named Micropack Pvt. Ltd.


