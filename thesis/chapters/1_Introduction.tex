\chapter{Introduction}
\begin{chapquote}
{Edward Witten}
``Particle physics is a modern name for the centuries old effort to understand the basic laws of physics.''
\end{chapquote}
Ever since its existence mankind is striving to unravel mysteries of nature around us. Several theoretical frameworks have been developed to answer key questions regarding our existence on this planet.  Continuous endeavours of the scientists in this regard lead to the evolution of a branch of physics, namely ``\textit{particle physics}'', dealing with the study of fundamental particles and the interactions among them. 

Standard Model (SM) of particle physics is a theoretical framework, developed in the 1960s and 70s, which encapsulates our current understandings about the particles and their interactions.
SM has been very successful in predicting the behaviour and certain characteristics of the elementary particles. It gives a compelling explanation of the existing experimental data.
The only missing link, the long sought after Higgs boson, was found in 2012 at the CERN Large Hadron Collider (LHC), and its properties were measured in the following years, thus completing the picture of elementary particles predicted by the SM.
However, the SM still leaves some unexplained phenomena, such as neutrino oscillations, the reason behind the spontaneous symmetry breaking which is responsible for the mass generation to the elementary particles and solves that unitarity problem in vector-boson scattering, baryon asymmetry, why there is mass hierarchy?
After the discovery of Higgs Boson, the major goal of the physics programme at the LHC is to provide a definitive answer to these open questions by significantly expanding both the energy reach and the collision rate with respect to the previous experiments and its own Run periods.

A special effort is being made to investigate the possible structure of the $SU(2)\times U(1)$-breaking physics and its experimental signature. These includes the precision measurement in the electroweak sector~\cite{Baak2013} and to look for the vector boson scattering.

The scattering of the vector bosons is strongly connected to the electroweak symmetry breaking (EWSB) in the SM, regulated by the Brout-Englert-Higgs mechanism.
The electroweak gauge bosons acquire mass through the EWSB, turning the massless Goldstone bosons in their longitudinal polarization.
Recent discovery of the Higgs boson indicates that the ultimate EWSB should be similar to the existing Higgs mechanism which needs to be scrutinized.
The VBS process also provides information on the quartic vector boson interactions, a part of the SM which has not been tested yet and could be modified by the existence new physics.
Thus, the VBS  emerges as a strong candidate process to serve the two-fold purpose to perform the precision measurements and side-by-side look for the new physics phenomenon.  


This thesis is organized as follows:

Chapter~\ref{cha:standard_model__vector_boson_scattering}, begins with a brief introduction to the SM of particle physics, followed by a prelude to the main thesis topic i.e. triple or quartic gauge couplings. A mathematical framework is discussed to explain the generation of triple or quartic gauge couplings in the SM followed with the brief introduction of the Higgs mechanism. The experimental data used to study VBS processes were collected during the proton-proton collisions run at the LHC using the CMS detector. The design and operation of the LHC machine are reported in the Chapter~\ref{cha:the_lhc_and_cms_machine}. Different sub-detector components of the CMS detector and their working is also discussed. Chapter-\ref{cha:gas_electron_multiplier} is devoted to the hardware activities carried out for the CMS muon system upgrades, specially the GEM detectors. Different measurements performed during the beam test studies of the GEM detectors are also provided. The data analysis performed to study the current physics topic of anomalous quartic gauge couplings is described in Chapter-~\ref{cha:anomalous_quartic_gauge_coupling_measurement} and an interpertation of results is given at the end of the Chapter.