\chapter{Introduction}
\begin{chapquote}
{Edward Witten}
``Particle physics is a modern name for the centuries old effort to understand the basic laws of physics.''
\end{chapquote}
Ever since its existence mankind is striving to unravel mysteries of Nature around us. Several theoretical frameworks have been developed to answer key questions regarding our existence on this planet.  Continuous endeavours of the scientists in this regard lead to the evolution of a branch of physics, namely ``\textit{particle physics}'', dealing with the study of fundamental particles and the interactions among them. 

Standard Model (SM) of particle physics is a theoretical framework, developed in the 1960s and 70s, which encapsulates our current understandings about the particles and their interactions.
SM has been very successful in predicting the behaviour and certain characteristics of the elementary particles. It gives a compelling explanation of the existing experimental data.
The only missing link, the long sought after Higgs boson, was found in 2012 at the CERN Large Hadron Collider (LHC), and its properties were measured in the following years, thus completing the picture of elementary particles predicted by the SM.
However, the SM still leaves some unexplained phenomena, such as neutrino oscillations, the reason behind the spontaneous symmetry breaking which is responsible for the mass generation to the elementary particles and solves that unitarity problem in vector-boson scattering, baryon asymmetry, why there is mass hierarchy?
After the discovery of Higgs Boson, the major goal of the physics programme at the LHC is to provide a definitive answer to these open questions by significantly expanding both the energy reach and the collision rate with respect to 
the previous experiments and its own Run periods.
Now, one of important task is to investigate the possible structure of the $SU(2)\times U(1)$-breaking physics and its experimental signature. These includes the precision measurement in the electroweak sector~\cite{Baak2013} and to look for the vector boson scattering.

	
As, the scattering of the vector bosons is strongly connected to electroweak symmetry breaking (EWSB) in the SM, regulated by the Brout-Englert-Higgs mechanism.
These electroweak gauge bosons acquire mass through the EWSB, turning the massless Goldstone bosons in their longitudinal polarization.
The recent discovery of the Higgs boson indicates that the ultimate EWSB should be similar to  the existing Higgs mechanism(ref).
The VBS process also contains information on the quartic vector boson interactions, a part of the SM which has not been well tested yet and could be modified by new physics.
Thus, the VBS  emerges as a strong candidate process to serve the two-fold purpose to perform the precision measurements and side-by-side look for the new physics phenomenon.  


This thesis is organized as follows. In Chapter-2, the brief introduction of the SM and the generation of triple or quartic gauge coupling in the SM followed with the brief introduction of the Higgs mechanism. As we know that for the study of the particle physics one need an accelerator and a detector that can accelerate the particles and collides them to probe the physics at small scale, which will help us to understand the behaviour and interaction of particles. Thus the briefly the one of largest accelerator Large Hadron Collider and its one of main purpose detectors design and working principle is explained in Chapter-3. Furthermore, one need to upgrade the detector with time to maintain the physics requirement. In this thesis one of sub-system upgrade details are mentioned in Chapter-4, which includes the test beam studies and the test and characterisation of the GEM foil performed at the University of Delhi. In Chapter-5, the physics studies of the anomalous quartic gauge coupling is given along with the doubly charged Higgs.