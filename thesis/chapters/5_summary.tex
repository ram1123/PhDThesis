\chapter{Summary} % (fold)
\label{cha:summary_&_outlook}
This thesis covers two broader topics: a) The characterisation study of made in India Gas Electron Multiplier (GEM) foil using the electrical and optical characterisation along with the upgrade studies performed for the CMS muon system upgrade using GEM detectors, also known as GE1/1 upgrade. b) physics analysis work performed for the measurement of the anomalous Quartic-Gauge Coupling (aQGC). A brief summary of these task is given below.

On the hardware front, work performed for the upgrade studies of the CMS detector muon endcap system is reported. For the CMS muon endcap detector system upgrade, the Gas Electron Multiplier (GEM) detectors are proposed to be installed during the Long Shutdown-2 (2019-2020) period because of its excellent performance in the harsh running environment like LHC. To test the functionality of these GEM detectors, several beam tests were carried out to measure their properties and evaluate their performance in terms of spatial and timing resolution, cluster size and efficiency measurements. The outcome from these beam test campaigns and the data analysis for the GEM detectors are presented here. Also, the characterisation studies for the GEM foils developed in India for the first time are also described using its optical and electrical characterisation. It is found that there are only 0.13\% defects present in the foil and the leakage current is always found to be $<$10 nA, set by CERN quality control criteria.

For future upgrades of the CMS to cope-up with the high luminosity of the LHC includes the upgrade of muon systeam. The CMS GEM community started R\&D on the phase-II upgrade that includes the GE2/1 project and the ME0 project.

For the physics analysis, a search for anomalous electroweak production of $WW$, $WZ$, and $ZZ$ boson pairs in association with two jets in proton-proton collisions at $13$ TeV is reported. The data sample corresponds to an integrated luminosity of $35.9$ fb$^{-1}$ collected with the CMS detector in proton-proton collisions at $\sqrt{s}=13~\TeV$. The hadronically decaying $W/Z$ boson is reconstructed as one large-radius jet and the other boson decays leptonic. Constraints on the quartic vector boson interactions in the framework of dimension-eight effective field theory operators are reported. The reported values are the most stringent limmits reported by the CMS Collaboration. 

Also, a theoritical interpretation of the observed result is given using the Georgi-Machacek model. The exclusion limit on the production cross-section for the charged Higgs boson times the branching fraction at 95\% CL as a function of the mass of the charged Higgs boson are reported. The reoprted values improve the previous published limits from CMS by a factor of $\sim$10.

This is just the starting phase of quartic-gauge coupling studies and the VBS. Due to low cross-section of the VBS process, it is not yet probed. However with the full Run-2 (2016-2018) data collected by the CMS, which corresponds to $\sim$150 $fb^{-1}$. Thus, using the full Run-2 data it will definitly improve the aQGC limits by several factors along with it one might be able to see the SM VBS results at least more than 2-sigma. The Run-3 of the LHC will be much more exiciting for this kind of studies and it should definitely reflect some information about the EWSB mechanism.


% chapter summary_&_outlook (end)