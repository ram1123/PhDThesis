%\pdfbookmark[1]{Abstract}{Abstract}
\chapter*{Abstract}  
In the Standard Model (SM) of particle physics, masses for the particles are generated by the Higgs mechanism, which requires the existence of a spin-0 particle called the Higgs boson. In July 2012, a new Higgs like particle is discovered with mass 125-126 GeV at the LHC. This may be the long-sought Higgs boson of the standard model (SM), which was proposed in the 1960s, or one of the Higgs bosons beyond the SM. For example, super-symmetric theories, little-Higgs models, and other extended Higgs sector such as the Georgi-Machacek model all contain a multitude of neutral as well as charged Higgs bosons. Because the current data still contain large uncertainties that these various extensions of the SM cannot be confirmed or ruled out decisively. 

Thus, now it is important to test the Electroweak Symmetry Breaking (EWSB) mechanism rigorously via precision measurement of the properties related to Higgs like the various coupling of Higgs to bosons and fermions and to look for the Vector Boson Scattering (VBS). As at high energy ($\approx 1~TeV$) the VBS violates the unitarity if there is no Higgs boson. Thus, it is one of the most important studies to understand the EWSB. Still, there is not enough data to directly probe the VBS but it can be studied by measuring the quartic vertices. This thesis focuses the study on the anomalous Quartic Gauge Coupling (aQGC) measurement. 

The aQGC measurement was done using both channel WV and ZV (Here, V represents to both W or Z bosons) in association with the two forward jets in proton-proton collisions from CMS detector at 13 TeV. In WV channel W decays leptonic and V decays hadronically while in ZV channel Z decays leptonic and V decays hadronically. The events are selected by requiring two jets with large rapidity separation and di-jet invariant mass, one or two leptons (electrons or muons), and a W or Z boson decaying hadronically. The hadronically decaying W/Z boson is reconstructed as one large-radius jet. The constraints on the quartic vector boson interactions are reported in this thesis in the framework of dimension-eight effective field theory operators.

Also, a model interpretation was done with the same channel using the Georgi-Machacek model. This model allows the doubly and singly charged Higgs using the Higgs triplets. The main feature of this model is that it preserves the custodial symmetry and provides neutrino Majorana mass. The exclusion limits on the charged Higgs bosons $\sigma_\mathrm{VBF}(\PHpmpm) \, \mathcal{B}(\PHpmpm\to \PW\PW)$ and $\sigma_\mathrm{VBF}(\PHpm) \, \mathcal{B}(\PHpm\to \PW\Z)$ at 95\% confidence level as functions of $m(\PHpm)$ and $m(\PHpmpm)$, respectively, reported in this thesis.

On the hardware front, work performed for the upgrade studies of the CMS muon endcap system. For CMS muon system endcap upgrade the Gas Electron Multiplier (GEM) detector was proposed to install during the Long Shutdown-2 (2019-2020). To test the functionality of these GEM detectors the beam test was performed to measure its performance like space resolution, time resolution, cluster size and efficiency, with the muon beam having the energy of $\approx 150~GeV$. The University of Delhi in collaboration with other Indian institutes actively participated in these beam test campaigns and data analysis for GEM detectors. Also, the characterisation of GEM foil developed in Indian for the CMS upgrade is mentioned.
