%\pdfbookmark[1]{Abstract}{Abstract}
% \chapter*{Abstract} 
\begin{center}
  \textbf{\large{Abstract}}
\end{center} 
In the Standard Model (SM) of particle physics, masses of the particles are generated by the Higgs mechanism which require the existence of a spin-0 particle called the Higgs boson. In July 2012, a new Higgs-like particle, with mass $\approx$125 GeV, was discovered at the Large Hadron Collider (LHC). This might be the long-sought SM Higgs boson predicted in the 1960s, or one of the Higgs bosons predicted by the several beyond SM (BSM) scenarios. Several BSM scenarios, such as, super-symmetry, little-Higgs models, and others from the extended Higgs sectors such as the Georgi-Machacek model, contain a multitude of neutral as well as charged Higgs bosons. Till now the existing results contain large uncertainties, thus various extensions of the SM cannot be confirmed or ruled out precisely.
This necessitates to scrutinize, the ElectroWeak Symmetry Breaking (EWSB) mechanism rigorously by carrying out the precision measurements of the Higgs boson properties and the couplings of the electroweak vector bosons (W and Z) with the Higgs boson via the Vector Boson Scattering (VBS) processes.

In the absence of the Higgs boson, the VBS processes violate the unitarity at an energy scale $\approx$1 TeV. Thus, this is one of the most important studies that could help us to understand the EWSB mechanism. Due to statistical constraints, VBS could be probed indirectly by measuring the quartic vertices. This thesis is based on the study of the anomalous Quartic Gauge Coupling (aQGC) processes using the proton-proton collision data at a centre-of-mass energy of 13 TeV with an integrated luminosity of $35.9~fb^{-1}$, collected using the Compact Muon Solenoid (CMS) detector at the CERN LHC in 2016.

This thesis presents the anomalous quartic gauge coupling (aQGC) measurement in the framework of dimension-eight effective field theory operators. It was performed using two channels: WV and ZV (where, V could be either a W or a Z boson) in association with the two jets produced in the forward pseudo-rapidity regions. For the WV (ZV) channels, only leptonic decay of W (Z) bosons are considered, while the V decays hadronically into a merged jet having large radii, with radius parameter 0.8. The constrains are imposed on the aQGC operators at 95\% confidence level (CL).

Furthermore, a theoretical interpretation of the observed results is given using the Georgi-Machacek model. The exclusion limits on the production cross-section for the charged Higgs bosons times the branching fraction at 95\% CL as a function of the mass of the charged Higgs boson are reported in this thesis.

On the hardware front, work performed for the upgrade studies of the CMS detector's muon endcaps is reported. For the CMS muon endcap detector system upgrade, the Gas Electron Multiplier (GEM) detectors are proposed to be installed during the Long Shutdown-2 (2019-2020) period. To test the functionality of these GEM detectors, several beam tests were carried out in 2014, to measure their properties and evaluate their performance in terms of spatial and timing resolution, cluster size and efficiency measurements. The outcome from these beam test campaigns and the data analysis for the GEM detectors are presented here. Also, the characterisation studies for the GEM foils developed in India for the first time are also described in terms of its electrical and optical properties.
