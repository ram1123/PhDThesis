%\pdfbookmark[1]{Abstract}{Abstract}
\chapter*{Synopsis}
In the Standard Model (SM) of particle physics, masses for the particles are generated by the Higgs mechanism which requires the existence of a spin-0 particle called the Higgs boson. In July 2012, a new Higgs-like particle, with mass $\approx$125 GeV, was discovered at the Large Hadron Collider. This might be the long-sought SM Higgs boson predicted in the 1960s, or one of the Higgs bosons predicted by the several beyond the SM scenarios. Several beyond SM scenarios, such as, super-symmetry, little-Higgs models, and others from the extended Higgs sectors such as the Georgi-Machacek model, contain a multitude of neutral as well as charged Higgs bosons. Till now the existing results contain large uncertainties, thus various extensions of the SM cannot be confirmed or ruled out decisively.
This necessitates to scrutinize, the ElectroWeak Symmetry Breaking (EWSB) mechanism rigorously by carrying out the precision measurements of the Higgs boson properties and the couplings of the electroweak vector bosons (W and Z) with the Higgs boson via the Vector Boson Scattering (VBS) processes.

VBS processes violate the unitarity at an energy scale $\approx$1 TeV in the absence of the Higgs boson. Thus, it is one of the most important studies that could help us to understand the EWSB mechanism. Due to the statistical constraints, VBS could be probed indirectly by measuring the quartic vertices. This thesis is based on the study of the anomalous Quartic Gauge Coupling (aQGC) processes using the proton-proton collision data at a centre-of-mass energy of 13 TeV, collected using the Compact Muon Solenoid (CMS) detector at the CERN LHC.

The aQGC measurement is performed using two channels: WV and ZV (here, V could be either a W or a Z boson) in association with the two jets produced in the forward pseudo-rapidity regions. For the WV (ZV) channel, only leptonic decays of W (Z) bosons are considered, while the V decays hadronically into jets having large radii (having radius parameter 0.8). The events are selected by requiring two jets with large rapidity separation and di-jet invariant mass, one or two leptons (electrons or muons). Constraints are imposed on the quartic vector boson interactions in the framework of dimension-eight effective field theory operators at 95\% confidence level (CL).

Furthermore, a theoretical interpretation of the observed results is given using the Georgi-Machacek model. This model predicts the existence of doubly and singly charged Higgs bosons using the Higgs triplets. The main feature of this model is that it preserves the custodial symmetry and provides neutrino with a Majorana mass. The exclusion limits on the production cross-section for the charged Higgs bosons times the branching fraction at 95\% CL as a function of the mass of the charged Higgs boson are reported in this thesis.

% The aQGC measurement was done using both channel WV and ZV (Here, V could be both W or Z bosons) in association with the two forward jets in proton-proton collisions from CMS detector at 13 TeV. In WV channel W decays leptonic and V decays hadronically while in ZV channel Z decays leptonic and V decays hadronically. The events are selected by requiring two jets with large rapidity separation and di-jet invariant mass, one or two leptons (electrons or muons), and a W or Z boson decaying hadronically. The hadronically decaying W/Z boson is reconstructed as one large-radius jet. This thesis represents the constraints on the quartic vector boson interactions in the framework of dimension-eight effective field theory operators.

% Furthermore, this thesis reports, a model interpretation was done with the same channel using the Georgi-Machacek model. This model allows the doubly and singly charged Higgs using the Higgs triplets. The main feature of this model is that it preserves the custodial symmetry and provides neutrino Majorana mass. The exclusion limits on the charged Higgs bosons $\sigma_\mathrm{VBF}(\PHpmpm) \, \mathcal{B}(\PHpmpm\to \PW\PW)$ and $\sigma_\mathrm{VBF}(\PHpm) \, \mathcal{B}(\PHpm\to \PW\Z)$ at 95\% confidence level as functions of $m(\PHpm)$ and $m(\PHpmpm)$, respectively, reported in this thesis.

On the hardware front, work performed for the upgrade studies of the CMS detector's muon endcap system is reported. For the CMS muon endcap detector system upgrade, the Gas Electron Multiplier (GEM) detectors are proposed to be installed during the Long Shutdown-2 (2019-2020) period. To test the functionality of these GEM detectors, several beam tests were carried out to measure their properties and evaluate their performance in terms of spatial and timing resolution, cluster size and efficiency measurements. I actively participated in these beam test campaigns and also during the data analysis for the GEM detectors. Also, the characterisation studies for the GEM foils developed in India for the CMS upgrade are also described.

% These GEM detectors were bombarded with a 150 GeV muon beam.
% performed to measure its performance like space resolution, time resolution, cluster size and efficiency, with the muon beam having the energy of $\approx 150~GeV$. I actively participated in these beam test campaigns and data analysis for GEM detectors. Also, the characterisation of GEM foil developed in Indian for the CMS upgrade is mentioned.
This thesis is organized in five Chapters. A brief description of each of these chapters is provided below:

\textbf{Chapter 1} begins with a brief introduction to the Standard Model (SM) of particle physics, followed by a prelude to the main thesis topic i.e. triple or quartic gauge couplings. A mathematical framework is discussed to explain the generation of triple or quartic gauge couplings in the SM followed with the brief introduction to the Higgs mechanism and the anomalous triple and quartic gauge interactions based on the Effective Field Theory (EFT) approach. Finally, the chapter concluded with a discussion on the doubly charged Higgs model, i.e., Georgi-Machacek model.

\textbf{Chapter 2} discusses briefly the experimental apparatus used to accumulate the data for the physics studies reported in this thesis. This contains a description of the Large Hadron Collider (LHC) and its one of the two general main purpose detector i.e. the CMS detector. Also, different sub-detector components of the CMS detector and their working mechanisms are outlined. The trigger system being used in the CMS detector for the data collection is also discussed.

\textbf{Chapter 3} is devoted to the hardware activities carried out for the CMS muon detector system upgrade. Starting from a brief history of the gaseous detectors, the focus is shifted to the Gas Electron Multiplier (GEM) detectors for the CMS GE1/1 upgrade. The proposed design and configuration of these GEM detectors are discussed along with their working principle. Different measurements performed during the beam test studies of the GEM detectors are also given.

\textbf{Chapter 4} reports the various steps and procedures followed in the analysis starting from the Monte-Carlo (MC) event generation of the signal sample (i.e. $pp \rightarrow WWjj$) and one of the irreducible QCD initiated background with two vector bosons along with the two jets. This Chapter presents the different aspects of the data analysis to perform the aforementioned measurements. Starting from a description of the signal and background samples and the event selection criteria, this Chapter also includes the background estimation techniques adopted for this analysis. Finally, the obtained results are given and discussed in the last section of the Chapter.

\textbf{Chapter 5} provides a summary of the work done during this PhD thesis and a future prospect of the physics and hardware analyses.