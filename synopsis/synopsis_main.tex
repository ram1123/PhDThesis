%\pdfbookmark[1]{Abstract}{Abstract}
\chapter*{Synopsis}
In the Standard Model (SM) of particle physics, masses of the particles are generated by the Higgs mechanism which require the existence of a spin-0 particle~\cite{Englert1964,Higgs1964,Higgs1964a,Guralnik1964,Higgs1966,Kibble1967} called the Higgs boson. In July 2012, a new Higgs-like particle, with mass $\approx$125 GeV, was discovered at the Large Hadron Collider (LHC)~\cite{Chatrchyan:2012xdj,Aad:2012tfa}. This might be the long-sought SM Higgs boson predicted in the 1960s, or one of the Higgs bosons predicted by the several beyond the SM scenarios. Several beyond SM scenarios, such as, super-symmetry~\cite{Martin1997,Djouadi2005a}, little-Higgs models~\cite{Han2003}, and others from the extended Higgs sectors such as the Georgi-Machacek model~\cite{GEORGI1985463}, contain a multitude of neutral as well as charged Higgs bosons. Till now the existing results contain large uncertainties, thus various extensions of the SM cannot be confirmed or ruled out precisely.
This necessitates to scrutinize, the ElectroWeak Symmetry Breaking (EWSB) mechanism rigorously, by carrying out the precision measurements of the Higgs boson properties and the couplings of the electroweak vector bosons (W and Z) with the Higgs boson via the Vector Boson Scattering (VBS) processes.

In the absence of the Higgs boson, the VBS processes violate unitarity at an energy scale $\approx$1 TeV. Thus, this is one of the most important studies that could help us to understand the EWSB mechanism. Due to statistical constraints, VBS could be probed indirectly by measuring the quartic vertices. In this thesis anomalous Quartic Gauge Coupling (aQGC) measurement is reported using the model independent way using the Effective Field Theory (EFT) by parametrizing the effects of high energy on the energy scale available to us~\cite{aqgc_operators}. The new effective Lagrangian using the EFT is give as:
\begin{equation}
	\mathcal{L}_{eff} = \mathcal{L}_{SM} + \sum_{i=www,w,B, \phi W, \phi B} \frac{c_i}{\Lambda^2} {\mathcal{O}}_i + \sum_{j=0,1}\frac{f_{S,j}}{\Lambda^4} \mathcal{O}_{S,j} + \sum_{j=0,...,9}\frac{f_{T,j}}{\Lambda^4} \mathcal{O}_{T,j}  + \sum_{j=0,...,7} \frac{f_{M,j}}{\Lambda^4} \mathcal{O}_{M,j}
\end{equation}
Where, $\Lambda$ is the scale of new physics, the parameters $c_i$, $f_{S,j}$, $f_{T,j}$ and $f_{M,j}$ are the dimension less coupling-strength coefficient typically of $\mathcal{O}(1)$. In the above equation the dimension eight operators have only quartic couplings. There are total of 18 independent parameters that are shown in Table-\ref{table:aQGC_alloperator} out of them we measure 9 parameters, they are: $f_{S,0}$, $f_{S,1}$, $f_{M,0}$, $f_{M,1}$, $f_{M,6}$, $f_{M,7}$, $f_{T,0}$, $f_{T,1}$ and $f_{T,2}$. 
\begin{table}
\centering
% \begin{tabular}[!htbp]{|p{1.8cm} | c  |c  |c  |c  |c  |c  |c | c |c |}
{\scriptsize
\begin{tabular}[!htbp]{|l | c  |c  |c  |c  |c  |c  |c | c  |c |}
\hline
 Parameters   & WWWW & WWZZ & ZZZZ & WWAZ & WWAA & ZZZA & ZZAA & ZAAA & AAAA \\
\hline
$\bm{f_{S,0}}$, $\bm{f_{S,1}}$ &$\bm{\times}$ & $\bm{\times}$&$\bm{\times}$ & & & & & & \\
\hline
$\bm{f_{M,0}}$, $\bm{f_{M,1}}$, $\bm{f_{M,6}}$, $\bm{f_{M,7}}$  &$\bm{\times}$ &$\bm{\times}$ &$\bm{\times}$ &$\bm{\times}$ &$\bm{\times}$ &$\bm{\times}$ &$\bm{\times}$ & & \\
\hline
$f_{M,2}$, $f_{M,3}$, $f_{M,4}$, $f_{M,5}$  & &$\times$ &$\times$ &$\times$ &$\times$ &$\times$ &$\times$ & & \\
\hline
$\bm{f_{T,0}}$, $\bm{f_{T,1}}$, $\bm{f_{T,2}}$ &$\bm{\times}$ &$\bm{\times}$ &$\bm{\times}$ &$\bm{\times}$ &$\bm{\times}$ &$\bm{\times}$ &$\bm{\times}$ &$\bm{\times}$ &$\bm{\times}$ \\
\hline
$f_{T,5}$, $f_{T,6}$, $f_{T,7}$ & &$\times$ &$\times$ &$\times$ &$\times$ &$\times$ &$\times$ &$\times$ &$\times$ \\
\hline
$f_{T,8}$, $f_{T,9}$  & & &$\times$ & & &$\times$ &$\times$ &$\times$ &$\times$ \\
\hline
\end{tabular}
\caption{Quartic vertices modified by the different operators are marked with $\times$. In the first row W, Z and A refers to the W-boson, Z-boson and photon respectively. In the first column the bold parameters are measured and the limits are reported.}
\label{table:aQGC_alloperator}}
\end{table}
%
This measurement was done using the proton-proton collision data at a centre-of-mass energy of 13 TeV with an integrated luminosity of $35.9~fb^{-1}$, collected using the Compact Muon Solenoid (CMS) detector at the CERN LHC in 2016.

The aQGC measurement is performed using two channels: WV and ZV (where, V could be either a W or a Z boson) in association with the two jets produced in forward pseudo-rapidity regions. For the WV (ZV) channels, only leptonic decays of W (Z) bosons are considered, while the V decays hadronically into a merged jet having large radii (having radius parameter 0.8). The events are selected by requiring two jets at large rapidity separation having large di-jet invariant mass, one or two leptons (electrons or muons), a fat jet with large radii and missing transverse momentum. Constraints are imposed on the quartic vector boson interactions in the framework of dimension-eight effective field theory operators at 95\% confidence level (CL).

The WV and ZV (specially, the $W^\pm W^\pm$ and $W^\pm Z$) fusion channels are also of interest because they are involved during the production and decay of a heavy, singly or doubly charged Higgs boson. Thus, these channel also provide us the means of study the couplings of type $W^\pm Z H^\pm$ and $H^{\pm \pm}W^\pm W^\pm$~\cite{Vega1990}. 

% These couplings may appear, unsuppressed, in extensions of the standard model which contain Higgs representations higher than a doublet representation[5,6]. If the charged Higgs are heavy, that is if ffiH > 2mw, then these couplings may dominate their interactions. The details and motivations for these models will not be discussed here3 . This paper only attempts to answer the question, "what if" such couplings exist and are not suppressed. In particular, the prospects for detection of a heavy (MH > 2mw) doubly or singly charged scalar are discussed. This discussion must necessarily be within the context of the w+w+ and W-Z fusion processes.
%
% Furthermore, a theoretical interpretation of the observed results is given using the Georgi-Machacek model~\cite{GEORGI1985463}. 
In this thesis the singly and doubly charged Higgs are considered in the framework of a specific model suggested by Georgi and Machacek~\cite{GEORGI1985463}. This model predicts the existence of doubly and singly charged Higgs bosons using the Higgs triplets. The main feature of this model is that it preserves the custodial symmetry and provides neutrinos with a Majorana mass. 
In this model the strength of couplings of charged Higgs with the vector bosons are parameterized using $sin(\theta_H)$, where $sin(\theta_H)=0$ will corresponds to the SM scenario. The measurement of $sin(\theta_H)$ will reflect the extent to which triplet scalar representation participates in the EWSB.
% The parameter $sin(\theta_H)$ measures the extent to which triplet scalar representations participate in the EWSB.
%
% In this model the strength of the couplings w+w+ H-- and w+ ZH- are parameterized by sinOH, where sin(}H = 0 would correspond to the standard model. One can think of this parameter as a measure of the extent to which triplet scalar representations participate in the spontaneous symmetry breaking of the SU(2) X Uy(l) symmetry. In the Georgi-Machacel model the scalar masses and sin(}H remain unconstrained.
% A doubly charged Higgs would produce a detectable signature if its couplings to the gauge bosons are not suppressed. For example, for sinBH = .5, the peak in the invariant mass is about three times bigger than the standard model signal. This can be seen in fig. 2 were it is noticed that the width of the doubly charged Higgs decreases with sineH, but the height of the peak is practically independent of sinBH. The peak becomes more defined, while the area below the peak decreases as sin2 ()H decreases, i.e. the cross section due to just the s-channel Higgs decreases as sin2 eH.
% The total cross section as a function of mH for various values of sin BH is presented in fig. 3. For comparison, the contributions coming from only the H++ s-channel diagram is also shown. For SSC design parameters each picobarn corresponds to about 334 events per year where the ~V's decay into either e-v. or Jl-171-'.
%
%
The exclusion limits on the production cross-section for the charged Higgs bosons times the branching fraction at 95\% CL as a function of the mass of the charged Higgs boson and the excluded values of the $sin(\theta_H)$ as a function of charged Higgs mass are reported in this thesis.

On the hardware front, work performed for the upgrade studies of the CMS detector's muon endcaps is reported. For the CMS muon endcap detector system upgrade, the Gas Electron Multiplier (GEM) detectors are proposed to be installed during the Long Shutdown-2 (2019-2020) period due to its excellent performance in the harsh running environment like LHC. To test the functionality of these GEM detectors, several beam tests were carried out in 2014 to measure their properties and evaluate their performance in terms of spatial and timing resolution, cluster size and efficiency measurements. The outcome from these beam test campaigns and the data analysis for the GEM detectors are presented here. Also, the characterisation studies for the GEM foils developed in India for the first time are described.

This thesis is organized in five Chapters. A brief description of each of these chapters is provided below:

\textbf{Chapter 1} begins with a brief introduction to the Standard Model (SM) of particle physics, followed by a prelude to the main thesis topic, i.e. triple or quartic gauge couplings. A mathematical framework is discussed to explain the generation of triple/quartic gauge couplings in the SM, followed by the brief introduction of the Higgs mechanism and the anomalous triple and quartic gauge interactions based on the approach of Effective Field Theory (EFT). Finally, the chapter concludes with a discussion on the doubly charged Higgs model, i.e., Georgi-Machacek model.

\textbf{Chapter 2} discusses the experimental apparatus used to collect the data for the physics studies reported in this thesis. This contains a description of the Large Hadron Collider (LHC)~\cite{LHC-tdr-vol1,LHC-tdr-vol2,LHC-tdr-vol3} and its one of the two general purpose detectors, i.e. the CMS detector~\cite{paper:JINST:CMSCollaboration}. The LHC is the world's most powerful particle accelerator and collider, located inside a tunnel of 27 km circumference, about 100 m underground at Swiss-France border. Currently, the LHC is operating at 13 TeV center of mass energy with peak luminosity $\mathcal{L}~ \simeq ~2.1 \times 10^{34}~cm^{-2}s^{-1}$~\cite{cms-lumi-public-results,Muratori2006}. Also, different sub-detector components of the CMS detector and their working mechanisms are outlined. The trigger system being used in the CMS detector for the data collection is also discussed which consists of the two-tier trigger. The first level (L1) of the CMS trigger consists of custom hardware processes running synchronously with the LHC bunch crossing frequency of $\sim$40 $MHz$. This uses information from the calorimeters and the muon detectors only to select the most interesting events within the time interval of less than 4 $\mu s$. The second level trigger is known as the High-Level Trigger (HLT). This uses fast offline reconstruction algorithm that uses information from all sub-systems, i.e., calorimeters, muon system as well as tracker to decide to keep or reject the event. This step further decrease the event rate from around 100 $kHz$ to about 100 $Hz$, before data storage.

\textbf{Chapter 3} is devoted to the hardware activities carried out for the CMS muon detector system upgrade. Starting from a brief history of the gaseous detectors, the focus is shifted to the Gas Electron Multiplier (GEM) detectors which is one of the excellent gaseous detectors having unprecedented spatial resolution, larger sensitive/detection area, with higher rate capability and good operational stability over the longer operating periods. This detector is approved by the CMS collaboration for the upgrade of the CMS muon endcap system during Long Shut-down 2 (2019-2020). This upgrade project is named as GE1/1 upgrade, where the letter ``G'' stands for GEM, ``E'' stands for End-cap, the first ``1'' corresponds to the first muon station and the second ``1'' corresponds to the first ring of the station~\cite{Colaleo:2021453}. The proposed design and configuration of these GEM detectors are discussed along with their working principle. For observing the performance of the prototype of GEM in the real environment the GE1/1 detector was tested in several beam tests. This thesis described the details and results from the 2014 beam test in terms of measured efficiency, time resolution and cluster size~\cite{Abbaneo2015,Sharma2018}. In this beam test we tested the GE1/1 detector with gas mixtures of Ar \& $CO_2$ and also Ar, $CO_2$ \& $CF_4$. One of the important conclusions drawn from this study was that one can operate the GEM detector without using $CF_4$ gas, which is a non-eco friendly gas, without compromising the efficiency and the time resolution of the detector.

This chapter also contains the characterization studies of the GEM foil developed in India. An Indian company, Micropack Pvt. Ltd., got the technology for GEM foil production through the Transfer of Technology (TOT) agreement with CERN. It was successfully able to produce $10~cm~\times~10~cm$ foils using the double mask technique~\cite{DEOLIVEIRA2009}. This GEM foil is characterized using the optical and electrical method. These results are mentioned in terms of the defects and the leakage current in GEM foils that we observed.


\textbf{Chapter 4} reports the various steps and procedures followed in the analysis for the Anomalous Quartic Gauge Couplings (aQGC) measurement using the model independent way using the Effective Field Theory (EFT) for the dimension-eight operators with the data collected during 2016 (36 $fb^{-1}$) in proton-proton collision at 13 TeV by the CMS detector. Among all the allowed processes, the process $pp \rightarrow WV jj$ and $pp \rightarrow ZV jj$ is used for the study. Here V denotes W or Z boson and they are always allowed to decay hadronically. The other vector boson decays leptonically. The study starts by measuring the interference between the electroweak process $pp \rightarrow VV jj$ and the QCD initiated process. The study showed that we have less than 1\% interference between the two. The major background $W+jets$ was estimated in a data-driven way using the alpha-ratio method~\cite{WVaTGC2016,VV_resonance_2016}. The events are selected by requiring two jets with large rapidity separation having large di-jet invariant mass, one or two leptons (electrons or muons), and a boosted W or Z boson decaying hadronically. The hadronically decaying W/Z boson is reconstructed as one large radius jet having radius parameter 0.8. Finally, the limits on the dimension-eight operators are given using the frequentist approach in asymptotic approximation.
The exclusion limits on the production cross-section for the charged Higgs bosons times the branching fraction at 95\% CL as a function of the mass of the charged Higgs boson and the excluded values of the $sin(\theta_H)$ as a function of charged Higgs mass are also reported in this thesis.


\textbf{Chapter 5} provides a summary of the work done during this PhD thesis for the physics analysis and the hardware work on the GE1/1 upgrade along with lab development and the characterization of the Indian GEM foil at the University of Delhi. This thesis also discuss the future prospects of the physics and hardware upgrade.

